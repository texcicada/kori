
%\defbibentryset{cwamset}{cwam,ama}

\newcommand\itemmarkerdefault{\textcolor{blue}{O}}
\newcommand\setitemmarker{\itemmarkerdefault}
\newcommand\setitemmarkercb{\colorbox{blue!20}}


\makeatletter
\newcommand*{\begrelateddelimmultivolumeset}{\newunitpunct\par\nobreak}
\newbibmacro*{related:multivolumeset}[1]{%
  \entrydata*{#1}{%
    \ifboolexpr{%====================================
      test {\ifentrytype{book}}
    }
    {%
  		\printfield[bibentrysetcount]{entrysetcount}%
  		}
  		{%
%    \printtext{\colorbox{blue!20}{\printfield[bibentrysetcount]{entrysetcount}}}%%%
    \printtext{\setitemmarkercb{\setitemmarker}}%%%
        }
    \blx@anchor % this is to get the link right, everything else is
                % a straightforward extension of related:multivolume
    \setunit{\addspace}%
    \ifboolexpr{%====================================
      test {\ifentrytype{book}}
    }
    {
    \printtext{%
      \printfield{volume}%
      \printfield{part}}%
    \setunit*{\addcolon\space}%
		\usebibmacro{title}%
		}
		{
		\addspace%%%
		--%%%
    \textcolor{blue}{\usebibmacro{title}}%%%
    }
    \ifboolexpr{
      test {\ifnamesequal{author}{savedauthor}}
      or
      test {\ifnameundef{author}}
    }
      {}
      {\usebibmacro{bytypestrg}{author}{author}%
       \setunit{\addspace}%
       \printnames[byauthor]{author}
       \newunit\newblock}%
    \ifboolexpr{
      test {\ifnamesequal{editor}{savededitor}}
      or
      test {\ifnameundef{editor}}
    }
      {}
      {\usebibmacro{byeditor+others}%
       \newunit\newblock}%
    \ifboolexpr{
      test {\iflistsequal{location}{savedlocation}}
      or
      test {\iflistundef{location}}
    }
      {}
      {\printlist{location}}%<- typo
    \ifboolexpr{
      test {\iflistsequal{publisher}{savedpublisher}}
      or
      test {\iflistundef{publisher}}
    }
      {\setunit*{\addcomma\space}}
      {\setunit*{\addcolon\space}%
       \printlist{publisher}%
       \setunit*{\addcomma\space}}%
    \ifboolexpr{%====================================
      test {\ifentrytype{book}}
    }
    {
  \printdate%
		}
		{
		\printfield{pages}%%%
		}
  \newunit\newblock
  \ifboolexpr{
    test {\iffieldsequal{isbn}{savedisbn}}
    or
    not togl {bbx:isbn}
  }
    {}
    {\printfield{isbn}}}}

% {<entrykey>}
% get the hash-keys of all related 'children'
\newcommand*{\blx@fakeset@gethashkey}[1]{%
  \listcsxadd{blx@fakeset@hashkeys@toprocess@\the\c@refsection}{%
    \detokenize{#1}}%
  % save fakeset parent of hash-key
  \csxdef{blx@fakeset@setp@\the\c@refsection @#1}{\strfield{entrykey}}%
  \ifcsundef{blx@fakeset@plist@\the\c@refsection}
    {\global\cslet{blx@fakeset@plist@\the\c@refsection}\@empty}
    {}%
  \xifinlistcs{\strfield{entrykey}}{blx@fakeset@plist@\the\c@refsection}
    {}
    {\listcsxadd{blx@fakeset@plist@\the\c@refsection}{\strfield{entrykey}}}%
  \advance\blx@tempcnta\@ne
  \csxdef{blx@fakeset@seti@\the\c@refsection @#1}{\the\blx@tempcnta}%
}

% {<cmd>}{<csv field>}
% execute <cmd> for each item in the csv field
% there should be a way to avoid the \def here ...
\newrobustcmd*{\blx@fakeset@forcsvfield}[2]{%
  \def\blx@fakeset@forcsvfield@handler{#1}
  \expandafter\expandafter\expandafter\expandafter\expandafter\expandafter
  \expandafter\expandafter\expandafter\expandafter\expandafter\expandafter
  \expandafter\expandafter\expandafter\expandafter\expandafter\expandafter
  \expandafter\expandafter\expandafter\expandafter\expandafter\expandafter
  \expandafter\expandafter\expandafter\expandafter\expandafter\expandafter
  \expandafter
  \forcsvlist
  \expandafter\expandafter\expandafter\expandafter\expandafter\expandafter
  \expandafter\expandafter\expandafter\expandafter\expandafter\expandafter
  \expandafter\expandafter\expandafter\expandafter\expandafter\expandafter
  \expandafter\expandafter\expandafter\expandafter\expandafter\expandafter
  \expandafter\expandafter\expandafter\expandafter\expandafter\expandafter
  \expandafter
  \blx@fakeset@forcsvfield@handler
  \expandafter\expandafter\expandafter\expandafter\expandafter\expandafter
  \expandafter\expandafter\expandafter\expandafter\expandafter\expandafter
  \expandafter\expandafter\expandafter\expandafter\expandafter\expandafter
  \expandafter\expandafter\expandafter\expandafter\expandafter\expandafter
  \expandafter\expandafter\expandafter\expandafter\expandafter\expandafter
  \expandafter
  {\thefield{#2}}}

\AtDataInput{%
  \blx@tempcnta\z@
  \iffieldequalstr{relatedtype}{multivolumeset}
    {\iffieldundef{related}
       {}
       {\blx@fakeset@forcsvfield{\blx@fakeset@gethashkey}{related}}}
    {}%
  % sorted list of *all* entries in the .bbl including cloned entries
  \listcsxadd{blx@fakeset@dlist@aentry@\the\c@refsection
    @\blx@dlist@name}{\strfield{entrykey}}%
}

% resolve hash-key to bib-key and other usable stuff
% {<hash-key>}
\newcommand*{\blx@fakeset@resolvekeys}[1]{%
  % get the entrykey from the bib, not the cloned hash
  \entrydata{#1}{\xdef\blx@fakeset@temp@bibkey{\strfield{clonesourcekey}}}%
  % map hash-key to bib-key
  \csxdef{blx@fakeset@bibkey@\the\c@refsection @#1}{\blx@fakeset@temp@bibkey}%
  % map fakeset parent to list of its children
  \ifcsundef{blx@fakeset@setc@\the\c@refsection @%
             \csuse{blx@fakeset@setp@\the\c@refsection @#1}}
    {\global\cslet{blx@fakeset@setc@\the\c@refsection
                   @\csuse{blx@fakeset@setp@\the\c@refsection @#1}}
       \@empty
     \blx@tempcnta\z@}
    {}%
  \advance\blx@tempcnta\@ne
  \listcsxadd{blx@fakeset@setc@\the\c@refsection
              @\csuse{blx@fakeset@setp@\the\c@refsection @#1}}
    {\blx@fakeset@temp@bibkey}%
  % map bib-key to fakeset parent
  \csxdef{blx@setc@\the\c@refsection @\blx@fakeset@temp@bibkey}{%
    \csuse{blx@fakeset@setp@\the\c@refsection @#1}}%
  % copy data from hash-key to bib-key
  \global\csletcs{blx@data@\the\c@refsection @\blx@refcontext@context
                  @\blx@fakeset@temp@bibkey}%
    {blx@data@\the\c@refsection @\blx@refcontext@context @#1}%
  % fakeset count
  \iftoggle{blx@sortsets}
    {\csnumgdef{blx@fakeset@seti@\the\c@refsection @#1}{\blx@tempcnta}}
    {}%
  \global\csletcs{blx@seti@\the\c@refsection @\blx@fakeset@temp@bibkey}
    {blx@fakeset@seti@\the\c@refsection @#1}%
  \global\csletcs{blx@seti@\the\c@refsection @#1}
    {blx@fakeset@seti@\the\c@refsection @#1}%
}

\newcommand*{\blxfset@addrealkey}[2]{%
  \listgadd{#1}{#2}%
  \ifcsundef{blx@fakeset@setc@\the\c@refsection @#2}
    {}
    {\forlistcsloop{\listadd{#1}}{blx@fakeset@setc@\the\c@refsection @#2}}}

% {<list_1>}{<list_2>}{<item>}
% worker macro to sort:
% add <item> to <list_2> if it exists in <list_1>
\newcommand*{\blx@fakeset@sortbycsinto@i}[3]{%
  \xifinlist{\detokenize{#3}}{#1}
    {\listeadd{#2}{\detokenize{#3}}}
    {}}

% {<list_1>}{<cs list_2>}{<list_3>}
% sorts all entries in <list_2> in the order given by <cs list_2> into <list_3>
\newcommand*{\blx@fakeset@sortbycsinto}[3]{%
  \forlistcsloop{\blx@fakeset@sortbycsinto@i{#1}{#3}}{#2}}

\newcommand*{\blx@fakeset@csvstrtolist@i}[2]{%
  \listeadd{#1}{\detokenize{#2}}}

\newcommand*{\blx@fakeset@listtocsvstr@i}[2]{%
  \appto{#1}{#2,}}

\newcommand*{\blx@fakeset@listtocsvstr}[2]{%
  \forlistloop{\blx@fakeset@listtocsvstr@i{#1}}{#2}}

\newcommand*{\blx@fakeset@sortrelatedfield}[1]{%
  \begingroup
  \blx@getdata{#1}%
  \let\blx@fakeset@tempa\@empty
  \let\blx@fakeset@tempb\@empty
  \blx@fakeset@forcsvfield
    {\blx@fakeset@csvstrtolist@i{\blx@fakeset@tempa}}
    {related}%
  \blx@fakeset@sortbycsinto
    {\blx@fakeset@tempa}
    {blx@fakeset@dlist@aentry@\the\c@refsection @\blx@refcontext@context}
    {\blx@fakeset@tempb}
  \let\blx@fakeset@tempa\@empty
  \blx@fakeset@listtocsvstr{\blx@fakeset@tempa}{\blx@fakeset@tempb}%
  \blx@bbl@addentryfield{#1}{\the\c@refsection}{related}%
    {\blx@refcontext@context}{\blx@fakeset@tempa}%
  \endgroup
}

% we can only resolve our fakeset entries once all entries have been read
\preto{\blx@bbl@endrefsection}{%
  \letcs\blx@fakeset@tempa{blx@fakeset@hashkeys@toprocess@\the\c@refsection}%
  \ifundef\blx@fakeset@tempa
    {\let\blx@fakeset@tempa\@empty}
    {}%
  \iftoggle{blx@sortsets}
    {\letcs\blx@fakeset@tempb{blx@fakeset@dlist@aentry@\the\c@refsection
       @\blx@refcontext@context}%
     \blx@filtercitesort\blx@fakeset@tempb{blx@fakeset@tempa}%
     \let\blx@fakeset@tempa\blx@fakeset@tempb}
    {}%
  \forlistloop{\blx@fakeset@resolvekeys}\blx@fakeset@tempa
  \let\blx@fakeset@tempa\@empty
  \forlistcsloop
    {\blxfset@addrealkey{\blx@fakeset@tempa}}
    {blx@dlist@centry@\the\c@refsection @\blx@refcontext@context}%
  \global\cslet{blx@dlist@centry@\the\c@refsection @\blx@refcontext@context}%
    \blx@fakeset@tempa
  \iftoggle{blx@sortsets}
    {\ifcsundef{blx@fakeset@plist@\the\c@refsection}
       {}
       {\forlistcsloop
          {\blx@fakeset@sortrelatedfield}
          {blx@fakeset@plist@\the\c@refsection}}}
    {}%
}%
\makeatother

\DeclareSortingTemplate{bytitle}{
\sort{
		\field{sorttitle}
		\field{title}
}
}

%\DeclareRefcontext{rcbytitle}{sorting=bytitle}


\DefineBibliographyExtras{english}{%
  % d-m-y format for long dates
  \protected\def\mkbibdatelong#1#2#3{%
    \iffieldundef{#3}
      {}
      {\stripzeros{\thefield{#3}}%
       \iffieldundef{#2}{}{\nobreakspace}}%
    \iffieldundef{#2}
      {}
      {\mkbibmonth{\thefield{#2}}%
       \iffieldundef{#1}{}{\space}}%
    \iffieldbibstring{#1}{\bibstring{\thefield{#1}}}{\stripzeros{\thefield{#1}}}}%
  % d-m-y format for short dates
  \protected\def\mkbibdateshort#1#2#3{%
    \iffieldundef{#3}
      {}
      {\mkdatezeros{\thefield{#3}}%
       \iffieldundef{#2}{}{/}}%
    \iffieldundef{#2}
      {}
      {\mkdatezeros{\thefield{#2}}%
       \iffieldundef{#1}{}{/}}%
    \iffieldbibstring{#1}{\bibstring{\thefield{#1}}}{\mkdatezeros{\thefield{#1}}}}%
}




%\DeclareBibliographyAlias{newspaper}{article}


\DefineBibliographyExtras{english}{%
  % d-m-y format for long dates
  \protected\def\mkbibdatelong#1#2#3{%
    \iffieldundef{#3}
      {}
      {\stripzeros{\thefield{#3}}%
       \iffieldundef{#2}{}{\nobreakspace}}%
    \iffieldundef{#2}
      {}
      {\mkbibmonth{\thefield{#2}}%
       \iffieldundef{#1}{}{\space}}%
    \iffieldbibstring{#1}{\bibstring{\thefield{#1}}}{\stripzeros{\thefield{#1}}}}%
  % d-m-y format for short dates
  \protected\def\mkbibdateshort#1#2#3{%
    \iffieldundef{#3}
      {}
      {\mkdatezeros{\thefield{#3}}%
       \iffieldundef{#2}{}{/}}%
    \iffieldundef{#2}
      {}
      {\mkdatezeros{\thefield{#2}}%
       \iffieldundef{#1}{}{/}}%
    \iffieldbibstring{#1}{\bibstring{\thefield{#1}}}{\mkdatezeros{\thefield{#1}}}}%
}




%\DeclareBibliographyAlias{newspaper}{article}



%
%\newcommand\examplekey{amamv}
%\newcommand\docite[2]{{\sffamily\bfseries#1}: text\textcolor{red}{\csname #1\endcsname{#2}} text}
%\newcommand\dociteexample[1]{\docite{#1}{\examplekey}}


\newcommand\printbibliographymagazines{%
\begin{refcontext}{rcbytitle}
\printbibliography[type=periodical,
title=Magazines,
keyword=mv,
heading=subbibliography]
\end{refcontext}
\begin{refcontext}{rcnyt}
\printbibliography[type=article,
title=Articles,
keyword=mv,
heading=subbibliography]
\end{refcontext}
}